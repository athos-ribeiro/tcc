\chapter{Métricas}

\section{Métricas de Software}

Por volta dos anos 70, o desenvolvimento de software estava se tornando caótico. A demanda por novos softwares aumentava,
entretanto não existiam técnicas que apoiassem o desenvolvimento, com isso, os softwares desenvolvidos estavam passando
a ter uma qualidade cada vez pior, sendo qualidade o conjunto de propriedades do produto ou serviço, que lhe conferem 
aptidões para satisfazer as necessidades explícitas ou implícitas ~\cite{iso:8402}. Tendo isso em vista, o cenário de
desenvolvimento de software era comumente caracterizado da seguinte forma:

\begin{itemize}
  \item Cronograma e estimativas de custo grosseiramente imprecisas,
  \item Software de má qualidade, e
  \item Um índice de produtividade crescendo mais lentamente do que a demanda por software.
\end{itemize}

Esse cenário descrito acima foi referenciado como a "Crise do Software" ~\cite{arthur85}.

Para solucionar a "Crise do Software"  se via necessário um melhor gerenciamento do processo de desenvolvimento de software
e do software produzido, o método utilizado até então não se mostrava eficaz diante desse cenário caótico, as formas de medir 
e estimar não estavam atendendo as reais necessidades. Viu-se então as peculiaridades e complexidade do desenvolvimento de 
software. Levando em consideração que uma métrica é a composição de procedimentos para a definição de escalas e métodos para 
medidas ~\cite{iso:9126-1}, e a partir dessas dificuldades se definiu qual seria o real objetivo de uma métrica de software: 
identificação e medição dos parâmetros essenciais que afetam o desenvolvimento de software ~\cite{mills88}.

Portanto, as métricas de software possuem papel fundamental para o acompanhamento da qualidade tanto do processo de
desenvolvimento de software, quanto do produto. Para isso, é necessário a definição de boas métricas, o que facilitará
e servirá de insumo para um acompanhamento mais preciso. São consideradas boas métricas aquelas que facilitam a medição dos 
parâmetros de qualidade definidos para um determinado software ~\cite{mills88}. É importante tentar limitar a quantidade
de métricas utilizadas, até porque o tratamento de um grande volume de dados pode ser humanamente impossível 
~\cite{meirelles2013}, e para isso é importante definir e justificar os critérios adotados para a escolha das métricas.
É interessante que as métricas possuam as seguintes características \cite{fenton&pfleenger98,mills88}:

\begin{itemize}
  \item o que a métrica se propõe a medir deve ser claro;
  \item a métrica deve ser formalmente definida, seu valor deve estar atrelado ao objeto medido, independente de quem (ou 
    qual ferramenta) o obtenha;
  \item deve ser possível obter seu valor rapidamente e a baixo custo;
  \item a métrica deve medir efetivamente o proposto por ela;
  \item pequenas mudanças no software, por exemplo, não podem causar grandes mudanças no valor obtido;
  \item deve haver formas de mapeamento das métricas para o entendimento do comportamento das entidades analisadas através 
    da manipulação dos números obtidos e
  \item o resultado da métrica deve ser independente da linguagem de programação utilizada.
\end{itemize}

Além disso, as métricas de software podem ser classificadas quanto ao âmbito da sua aplicação, quanto ao critério utilizado 
na sua definição e quanto ao método de obtenção da medida ~\cite{meirelles2013}. Em resumo, uma métrica de software deve
ser válida, confiável e prática.

As métricas de software são divididas em algumas categorias. Elas podem ser divididas pelo método de obtenção, onde podem
ser \textit{primitivas} ou \textit{compostas}; método como serão determinadas, onde podem ser \textit{objetivas} ou 
\textit{subjetivas}; e também pode variar a sua escala, onde podem ser \textit{nominal}, \textit{ordinal}, \textit{intervalo}
ou \textit{racional}.

As métricas primitivas podem ser obtidas a partir da própria entidade, por leitura direta, possuem apenas uma medida, já as
métricas compostas são calculadas a partir de uma ou mais medidas, sendo essas métricas primitivas, logo, as métricas compostas
são derivadas de métricas primitivas. 

A forma de determinar uma métrica pode depender da avaliação de quem está interpretando-a e não somente dos dados em si, nesse
caso, chama-se de métrica subjetiva, dependendo de quem está avaliando o resultado final pode diferenciar ou não. Agora quando
existem regras claras e bem definidas para determinar a valoração de uma métrica, chamamos a mesma de métrica objetiva, não
dependendo da opinião de terceiros.

As metricas de software precisam ser coletadas em um modelo de dados específico que pode envolver cálculos ou análise 
estatística subjetiva ~\cite{meirelles2013}. Segundo \cite{fenton&pfleenger98}, existem quatro tipos de dados de medidas
reconhecidos para métricas de software. Abaixo serão apresentados os quatro tipos de escalas de métricas de software 
~\cite{meirelles2013}:

\begin{itemize}
  \item \textbf{Nominal}: existe um nome ou um valor para um atributo; no entanto, a ordem dos valores não tem nenhum 
    significado para a sua interpretação;
  \item \textbf{Ordinal}: os resultados estão em uma determinada ordem (ascendente ou descendente), mas a distância entre os 
    pontos dessa escala não tem significado;
  \item \textbf{Intervalo}: preserva a importância da ordem dos resultados e possui informações sobre o tamanho dos intervalos
    que separaram os pontos da escala, mas a relação entre os valores não são necessariamente válidas.
  \item \textbf{Racional}: semelhante a escala de intervalo, mas representando também as proporções entre as entidades e
    possuindo um zero absoluto. 
\end{itemize}

Geralmente, a escala racional é preferível as outras, já que a mesma possibilita de maneira mais eficiente a aplicação de 
operações matemáticas ~\cite{meirelles2013}.

No momento de definição de uma métrica de software é necessário pensar na classificação da mesma levando em consideração
todas as dimensões apresentadas, deve-se definir a forma de obtenção, valoração e escala.

Neste trabalho, serão trabalhadas em geral com métricas objetivas; primitivas e derivadas; e racional. Tendo em vista que o 
foco são métricas de código fonte que serão coletadas de forma automatizada por ferramentas especializadas.

\section{Métricas de Design de Código Fonte}

Métricas de design de software devem ser compreendidas como um conjunto de métricas que medem atributos do código-fonte que 
permitam a avaliação de produtos de software ~\cite{arthur&carlos2014}. A utilização correta de métricas de design no ciclo
de desenvolvimento de software oportuniza além da aferição da qualidade do produto, a refatoração. Na refatoração, a forma que
foi implementada determinada funcionalidade é repensada e reimplementada, de forma que melhore alguns atributos dos software,
como manutenibilidade, testabilidade, a própria qualidade interna do software. Abaixo serão apresentadas as principais métricas
de design que são possíveis de obter com o uso de ferramentas automatizadas para análise estática de código fonte 
~\cite{arthur&carlos2014,meirelles2013}:

\begin{itemize}
  \item \textbf{LOC (Lines of Codes - Linhas de Código)}: LOC calcula o número de linhas executáveis, desconsiderando linhas 
    em branco e comentários. Esta é a métrica de tamanho mais comum. Entretanto, deve ser cuidadosamente usada e composta, pois
    os parâmetros de comparação não devem ser os mesmos quando se varia a linguagem e estilo de programação.
  \item \textbf{(Total Number of Modules or Classes - Número Total de Módulos ou Classes)}: Esta métrica mensura o 
    tamanho do software baseados na quantidade de módulos e classes, sendo menos sensível por linguagens de programação, nível 
    de desenvolvedores e estilo de codificação.
  \item \textbf{AMLOC (Average Method LOC - Média de Número de Linhas de Código por Método)}: Esta métrica avalia a distribuição
    do código entre os métodos, sendo uma importante indicador de coesão, reutilização e outras características importantes. 
    Entretanto, assim como a LOC, deve ser avaliada considerando-se a linguagem e estilo de programação.
  \item \textbf{NOA (Number of Attributes - Número de Atributos)}: NOA calcula o número de atributos de uma classe, sendo 
    bastante importante para avaliar a coesão de uma classe.
  \item \textbf{(Total Number of Modules or Classes - Número Total de Módulos ou Classes)}: Esta métrica mensura o tamanho do 
    software baseados na quantidade de módulos e classes, sendo menos sensível por linguagens de programação, nível de
    desenvolvedores e estilo de codificação.
  \item \textbf{NOM (Number of Methods - Número de Métodos)}: Esta métrica se refere ao tamanho de uma classe medindo a 
    quantidade de operações de uma classe. Sua interpretação pode ser complicada. Um número excessivo de métodos pode 
    representar falta de coesão e de potencial de reusabilidade da classe. Por outro lado, pode representar uma classe bem 
    estruturada com operações bem definidas. Entretanto, a avaliação isolada desta métrica não permite este tipo de afirmação.
  \item \textbf{NPA (Number of Public Attributes - Número de Atributos Públicos)}: NPA mede basicamente o encapsulamento de 
    uma classe. Independente da linguagem, é desejado que este valor seja o mais baixo possível, pois é recomendado que a 
    manipulação de atributos de uma classe sejam realizados por métodos de acesso e operacionais.
  \item \textbf{NPM (Number of Public Methods - Número de Métodos Públicos)}: Esta métrica é muito importante para a 
    compreensão da abstração da classe, pois mede diretamente o tamanho da interface de acesso à mesma. NPM pode ser melhor
    utilizada para compreender o potencial de reusabilidade e coesão de uma classe do que a métrica NOM isoladamente.
  \item \textbf{ANPM (Average Number of Parameters per Method - Média de Parâmetros por Método)}: Essa métrica calcula a 
    média de parâmetros dos métodos da classe, onde não se deseja um valor alto.
  \item \textbf{MNPM (Maximum Number of Parameters per Method - Número Máximo de Parâmetros por Método)}: Essa métrica 
    corresponde à maior ocorrência de número de parâmetros dos métodos de uma classe.
  \item \textbf{DIT (Depth of Inheritance Tree - Profundidade da Árvore de Herança)}: Está métrica consiste no número de 
    classes ancestrais da classe em análise, sem considerar heranças provindas de frameworks ou bibliotecas. Para linguagens 
    com herança múltipla, o valor desta métrica é o DIT da maior hierarquia.
  \item \textbf{NOC (Number of Children - Número de Filhos)}: NOC consiste no número de filhos direto de uma classe.
  \item \textbf{ACCM (Average Cyclomatic Complexity per Method - Média da Complexidade Ciclomática por Método)}: Esta métrica 
    mede a complexidade média dos métodos de uma classe, baseando-se na complexidade dos fluxos de controle existente no método.
  \item \textbf{RFC (Response For a Class - Resposta de uma Classe)}: Esta métrica mede a complexidade de uma classe contando 
    o número de métodos que um objeto de uma classe pode invocar, tanto métodos internos quanto de outras classes.
  \item \textbf{ACC (Afferent Connections per Class - Conexões Aferentes por Classe)}: Esta métrica mede a conectividade de 
    uma classe a partir da contagem de quantas classes do sistema acessam um atributo ou método da classe em análise. Caso o
    valor de ACC de uma classe seja alto, modificações em sua estrutura podem afetar mais classes.
  \item \textbf{CBO (Coupling Between Objects - Acoplamento Entre Objetos)}: Esta métrica calcula de quantas classes a classe 
    em análise depende, sendo a recíproca da ACC.
  \item \textbf{COF (Coupling Factor - Fator de Acoplamento)}: Esta métrica é a razão entre o número acoplamento existente que 
    não sejam provindos de herança e do número total de possíveis acoplamentos. O máximo de acoplamento possível acontece 
    quando todas as classes estão e são acopladas com as outras classes do projeto.
  \item \textbf{LCOM4 (Lack of Cohesion in Methods - Ausência de Coesões em Métodos)}: Esta métrica cálcula o número de 
    componentes conectados em uma classe. Um componente conectado consiste em um conjunto de métodos relacionados. Dois
    métodos são relacionados se ambos acessam as mesmas variáveis da classe ou um método invoca ao outro.
\end{itemize}

Acima foram apresentadas métricas de design atreladas ao tamanho, coesão e acoplamento e atributos estruturais, que dão uma
visão geral sobre a qualidade do design projetado e implementado.

As métricas de design possuem relevância nesse trabalho pelo fato de que existem três causas básicas que podem surgir 
vulnerabilidades em um software, sendo elas erros de implementação, falhas de design e erros de configuração ou infraestrutura
do sistema ~\cite{malerba2010}. Portanto, as métricas de design possuem um papel importante para se evitar vulnerabilidades 
em um software.

Apesar de apresentadas neste trabalho algumas métricas de design, e a sua importância por ter associação direta com algumas 
vulnerabilidades, esse não é o foco do trabalho. O foco do estudo são as métricas de vulnerabilidade de código fonte que serão 
apresentadas na seção a seguir.

\section{Métricas de Vulnerabilidade de Código Fonte}

No mundo atual, onde praticamente todos estão interligados através da Internet, devemos nos preocupar cada vez mais com a 
segurança dos softwares utilizados. Uma simples falha de segurança em um software pode causar desde danos pessoais, como perca 
de arquivos pessoais, até mesmo perca de capital por parte de uma empresa multinacional. Sabendo que vulnerabilidade é um 
conjunto de condições que podem levar a violação de uma política de segurança explícita ou implícita 
~\cite{seacord&householder2005}, necessita-se, no contexto atual, de métricas de software voltadas especificamente para 
vulnerabilidades de código fonte, onde será possível monitorar de maneira mais clara e metódica o quão seguro é o software que 
está sendo desenvolvido ou analisado.

As métricas de vulnerabilidade de código fonte, se utilizadas adequadamente no processo de desenvolvimento de software, evitam
brechas para \textit{exploits} e ataques. Sendo \textit{exploit} um software ou uma técnica que tira vantagem de uma 
vulnerabilidade para violar uma política de segurança ~\cite{seacord&householder2005} e ataque uma série de passos intencionais 
seguidos por um atacante para violar uma política de segurança ~\cite{howard&longstaff98}. Um estudo de caso que exemplifica 
um cenário de não utilização de métricas de vulnerabilidade de código fonte e que esse fator foi preponderante para abrir
brechas para possíveis atacantes, foi o caso apresentado por ~\cite{aranha2012} no relatório sobre vulnerabilidades 
encontradas no software da urna eletrônica brasileira. Nesse caso, a utilização de ferramentas de análise estática de código
fonte voltadas para vulnerabilidades é essencial para que esse tipo de falha seja encontrada durante o período de 
desenvolvimento do software, e não depois da sua liberação e colocação em produção.

Tendo em vista a importância do monitoramento de métricas de vulnerabilidade de software, a seguir, serão apresentadas algumas
métricas dessa categoria voltadas principalmente para implementações utilizando as linguagens de programação C e C++:

\begin{itemize}
  \item \textbf{AN (Argument with 'nonnull' attribute passed null - Argumento com atributo não nulo passado nulo):} Ocorre
    quando uma função/método manipular um atributo sem valida-lo, acreditando que o mesmo não será nulo, e o argumento passado
    na chamada da função é nulo.
  \item \textbf{ASOM (Allocator sizeof operand mismatch - Operador incompatível com alocador sizeof)}: Ocorre quando o tipo
    da variável (por exemplo \textit{int}, \textit{char}, \textit{float}, \textit{double}) passado para a função 
    \textit{sizeof()} é inconsistente com o que foi passado para funções como \textit{malloc()}, \textit{calloc()} e 
    \textit{realloc()}.
  \item \textbf{AUV (Assigned value is garbage or undefined - Valor atribuído é lixo ou indefinido)}: Ocorre quando o valor 
    atribuído a uma variável é lixo ou indefinido.
  \item \textbf{BD (Bad deallocator - Desalocador ruim)}
  \item \textbf{BF (Bad free - Liberação ruim)}
  \item \textbf{DA (Dead assignment - Atribuição morta)}
  \item \textbf{DBZ (Divisions by zero - Divisão por zero)}
  \item \textbf{DF (Double free - Liberação dupla)}
  \item \textbf{DNP (Dereference of null pointer - Desreferencia de ponteiro nulo)}
  \item \textbf{DUPV (Dereference of undefined pointer value - Desreferencia ou indefinição do valor do ponteiro)}
  \item \textbf{FGBO (Potential buffer overflow in call to 'gets' - Potencial buffer overflow na chamada de 'gets')}
  \item \textbf{MLK (Memory leak - Vazamento de memória)}
  \item \textbf{OBAA (Out-of-bound array access - Acesso de array fora do limite)}
  \item \textbf{OSF (Offset free - Liberação com deslocamento)}
  \item \textbf{PITFC (Potential insecure temporary file in call 'mktemp' - Potencial insegurança de arquivo temporário na
    chamada 'mktemp')}
  \item \textbf{ROGU (Result of operation is garbage or undefined - Resultado da operação é lico ou indefinido)}
  \item \textbf{RSVA (Return of stack variable address - Retorno de endereço de uma variável da pilha)}
  \item \textbf{SAIGV (Stack address stored into global variable - Endereço da pilha armazendo em variável global)}
  \item \textbf{UA (Undefined allocation of 0 bytes - Alocação indefinida de 0 bytes)}
  \item \textbf{UAF (Use-after-free - Uso depois da liberação)}
  \item \textbf{UAV (Uninitialized argument value - Valor do argumento não inicializado)}
\end{itemize}

As métricas apresentadas acima são objetivas; primitivas e derivadas; e racionais. Lembrando que o valor dessas métricas
representam a quantidade de vezes que foram encontrados esses cenários de vulnerabilidade no código fonte do software em 
questão.

No decorrer do trabalho será apresentada a forma que se dará a definição dos valores de referência para acompanhamento dessas
métricas, já que, como foi apresentado, é essencial o bom monitoramento das mesmas. Os valores de referências servirão
de insumo para tomada de decisão após a extração das métricas, onde o engenheiro de software irá definir se o valor daquela
métrica em específico é aceitável ou não, e com isso, tomar ações corretivas se necessário.

