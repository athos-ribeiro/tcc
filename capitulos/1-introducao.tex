\chapter{Introdução} \label{cap:introducao}

A medição é um processo auxiliar essencial para o desenvolvimento de software
com qualidade. É importante medir para entender e controlar processos, produtos 
e projetos \cite{ministerio_processo2012}. Medidas fornecem informações sobre 
objetos (processos, produtos e projetos) e eventos, tornando-os compreensíveis 
e controláveis \cite{fenton&pfleenger98}. Entretanto, muitas vezes, mede-se
simplesmente por medir \cite{ministerio_processo2012}, não justificando todo o 
esforço gasto durante o processo.

O cenário apresentado acima é o que geralmente ocorre quando as pessoas não estão
capacitadas para trabalhar no dado contexto. Quando as pessoas não estão
preparadas para trabalhar com determinadas métricas de software existem duas opções:
utilizam de maneira inadequada, não agregando valor ao processo de desenvolvimento
de software; ou apenas não utilizam, mesmo que sejam importantes para o processo de
desenvolvimento.

Muitas vezes, as métricas de código fonte se encaixam bem nessa situação, não são 
utilizadas dentro do processo de desenvolvimento porque não se sabe o que deve ser
feito com as mesmas e que decisões tomar. Quando utilizadas, são utilizadas apenas
as métricas de código fonte mais conhecidas, como as 
métricas orientadas a objetos de \citeonline{chidamber&kemerer2002}, isso ocorre 
porque são métricas já consolidadas
e possuem várias diretrizes de como utiliza-las de maneira adequada. Outras classes 
de métricas de código fonte, como as métricas de vulnerabilidade de código fonte 
por exemplo, não são tão utilizadas no dia-a-dia do desenvolvimento de software, 
devido a falta de conhecimento de como interpreta-las e monitora-las.

É nesse contexto que se encaixa este trabalho, onde pessoas não utilizam ou utilizam
de maneira inadequada métricas de código fonte por falta de orientação de como 
interpreta-las e utiliza-las de maneira que agregue valor ao produto/negócio.
Dessa forma, o desenvolvimento de modelos preditivos para essa classe de métrica
facilitaria o seu entendimento e serviria de referência para tomadas de
decisões.

\section{Justificativa}

Sabendo que existe o problema dentro da Engenharia de Software de não saber como
interpretar e monitorar métricas de código fonte além das mais conhecidas, que são 
as métricas de \textit{design} de código fonte, espera-se conseguir auxiliar os 
Engenheiros de Software com este trabalho. Tentando mostrar a melhor forma de 
interpretar e acompanhar esses diferentes tipos de métricas que estão emergindo
nos últimos tempos, muitos não as utilizam devido a falta de conhecimento sobre
o que fazer com essas informações advindas dessas novas métricas.

Relacionado as métricas de vulnerabilidade de código fonte que é a primeira
classe de métricas de código fonte trabalhada neste trabalho, existe a
questão da segurança dos softwares, que está bastante em voga nos últimos
tempos, o que reforça a necessidade de inserção de métricas de vulnerabilidade
de código fonte dentro do ciclo de desenvolvimento de software. O encontro
dessas vulnerabilidades ainda dentro do ciclo de desenvolvimento facilita a
correção das mesmas, além de evitar custos de possíveis futuras manutenções
corretivas. Mas para isso, é necessário saber interpretar esses valores
e acompanhar no decorrer do ciclo de desenvolvimento.

Existem várias iniciativas de pesquisas relacionadas a como associar
características de \textit{design} do código fonte com vulnerabilidades, como
nos trabalhos de \citeonline{arthur&carlos2014} e
\citeonline{alshammari2009}; entretanto, não
existem muitas iniciativas no mundo acadêmico de como interpretar e acompanhar
os valores dessas métricas de ameaças de vulnerabilidade de código fonte, sendo
essa uma oportunidade ímpar de pesquisa.

Segundo um artigo divulgado pelo \textit{NIST} (\textit{National Institute of
Standards and Technology}), desenvolvido por \citeonline{jansen2009}, o tema
deste trabalho está totalmente alinhado com os pensamentos da agência norte
americana, onde a coleta de dados de projetos existentes e a análise dos mesmos
podem indicar padrões e informações importantes para a medição da segurança de
software, sendo essa apontada como uma possível área de pesquisa.

Dentre outras coisas, este trabalho também servirá de insumo para próximos
trabalhos, não só devido aos resultados obtidos, mas também a metodologia
utilizada..

\section{Objetivos} \label{sec:objetivos}

Levando em consideração o contexto apresentado anteriormente, o objetivo deste
trabalho é encontrar um modelo preditivo para métricas de ameaças de
vulnerabilidade de código fonte de baixa complexidade que sirva de referência
para outros projetos de software.

Para isso, outros objetivos paralelos precisam ser atingidos, como o
entendimento do processo de análise estática, principalmente o que diz respeito
a segurança de código fonte, como as ferramentas que realizam essa análise
funcionam, além de algum processo de análise estatística que auxilie no
desenvolvimento da pesquisa.

Em resumo, o principal objetivo de pesquisa deste trabalho é responder a
seguinte questão-problema:

\begin{center}
  \textit{É possível o desenvolvimento de um modelo preditivo de referência de
  baixa complexidade para acompanhamento e monitoramente de métricas de ameaças
de vulnerabilidade de código fonte em projetos de software?}
\end{center}


\section{Metodologia}

A metodologia que foi adotada para atingir o objetivo deste trabalho foi:

\begin{enumerate}
  \item Levantamento bibliográfico
    \begin{itemize}
      \item Análise estática de código fonte, principalmente voltada para a
        segurança do mesmo
      \item Taxonomia e classificação de vulnerabilidades de software
      \item Ferramentas de análise estática, com foco em seguraça do código
        fonte
      \item Análise exploratória de dados e alguns métodos e técnicas utilizados
    \end{itemize}

  \item Desenvolvimento de uma pesquisa empírica com o objetivo de se definir
    modelos preditivos para as métricas de ameaças de vulnerabilidade de código
    fonte selecionadas

  \item Identificar possíveis riscos elimitações relacionados a pesquisa
    desenvolvida e listar possíveis futuros trabalhos
\end{enumerate}

Mais detalhes sobre a metodologia adotada para o desenvolvimento da pesquisa
empírica apresentada anteriormente pode ser visto na seção
\ref{metodologia:planejamentopesquisa}.

\section{Organização do Trabalho}

Este trabalho está dividido em seis capítulos subsequentes. No capítulo
\ref{chap:analiseestatica}, são apresentados alguns conceitos como o de análise
estática e métricas de software, além de apresentar algumas peculiaridades sobre
a classificação e taxonomia de vulnerabilidades de software para em seguida
serem apresentadas as métricas de ameaças de vulnerabilidade de código fonte
utilizadas neste trabalho. No capítulo \ref{chap:ferramentas}, é apresentada a
importância de uma ferramente que automatize o processo de análise estática,
como essas ferramentas funcionam em geral e em seguida é aprofundado sobre cada
uma das etapas realizadas pelas ferramentas de análise estática de segurança de
código fonte. No capítulo \ref{eda}, é apresentado a diferença da análise
exploratória de dados para as abordagens estatísticas tradicionais, em seguida é
detalhado cada uma das etapas realizadas durante esse processo. No capítulo
\ref{metodologia}, contém toda a metodologia desenvolvida nesta pesquisa,
seguindo uma abordagem de análise exploratória de dados, para poder se definir
modelos preditivos para as métricas em questão. No capítulo
\ref{definicaomodelos}, são apresentados as definições e validações dos modelos
preditivos das métricas de ameaças de vulnerabilidade de código fonte,
continuando a abordagem estatística apresentada anteriormente, chegando ao final
com fórmulas matemáticas que representem-os. No capítulo \ref{chap:conclusao},
sendo esse o último capítulo do trabalho, serão feitas algumas conclusões,
respondendo a questão de pesquisa levantada, além de apresentar algumas
limitações e riscos que podem ameaçar este trabalho e uma lista de trabalhos
futuros.


