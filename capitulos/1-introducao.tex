\chapter{Introdução} \label{cap:introducao}

Introdução

\section{Justificativa}

\section{Objetivos} \label{sec:objetivos}

Levando em consideração o que foi apresentado anteriormente, este trabalho tem
como objetivo principal determinar uma forma de interpretar e acompanhar
métricas de código fonte além das métricas de \textit{design}. Nesta
etapa do trabalho, o objetivo é encontrar uma forma de interpretar e monitorar,
especificamente, métricas de vulnerabilidade de código fonte.

No entanto, para isso, espera-se se aprofundar mais na teoria das métricas de
software e cada tipo de métricas de código fonte que deseja-se estudar, com o 
intuito de dominar o contexto que está sendo trabalhado e suas nuâncias.

Com o conhecimento teórico em mãos parte-se para o conhecimento prático,
empírico. Tendo assim, como objetivo, realizar estudos empíricos utilizando 
projetos de software livre, as métricas em questão e ferramentas para 
automatização desse processo também livre, para tentar responder a questão de 
como se deve interpretar e acompanhar os valores dessas métricas de código fonte.
Com esses estudos empíricos é passível de encontrar várias informações relevantes
que ainda não foram pensadas no âmbito teórico. Além da necessidade de prática
em um trabalho de final de curso de Engenharia de Software.

Passando para as ferramentas utilizadas nos estudos empíricos, espera-se que
seja contribuição deste trabalho a manutenção evolutiva da ferramenta
\emph{Analizo}, com o que diz respeito a extração das métricas de
vulnerabilidade de código fonte. Esta funcionalidade já existe na ferramenta,
entretanto, necessita de alguns ajustes. Sendo o autor deste trabalho um dos
mantenedores da ferramenta e participante da equipe de desenvolvimento dessa
funcionalidade, espera-se contribuir com o projeto ao final, para que esta seja
a ferramenta para extração desse tipo de métrica de código fonte.

Em outra etapa do trabalho será envolvido a questão do aprendizado de máquina, e
será um dos objetivos tentar modelar a problemática da interpretação e
acompanhamento das métricas de código fonte, além das de \textit{design},
seguindo essa abordagem. Sendo escopo deste trabalho apenas a modelagem do
problema seguindo a abordagem de aprendizado de máquina e não a implementação
deste modelo.

Em resumo, o principal objetivo de pesquisa deste trabalho é responder a
seguinte questão:

\begin{itemize}
  \item Como deve-se interpretar e acompanhar as métricas de código fonte além 
    das métricas de \textit{design}?
\end{itemize}


\section{Metodologia}

A metodologia utilizada para se atingir os objetivos deste trabalho foram: uma
revisão bibliográfica acerca do assunto de métricas de software, métricas de
\textit{design} de código fonte e métricas de vulnerabilidade de código fonte,
possivelmente serão realizados estudos sobre novos tipos de métricas de código
fonte que serão adicionados a este trabalho; além da realização de um estudo de 
caso, baseado em \emph{Meirelles} (\citeyear{meirelles2013}), para tentar entender 
melhor as métricas de vulnerabilidadede código fonte, como as mesmas se comportam 
e qual a melhor forma de monitora-las, tendo em vista que ainda não existe uma
quantidade considerável de referencial teórico acerca do assunto. Após a
consolidação de como interpretar e monitorar essas métricas, este trabalho
partirá para a abordagem de aprendizado de máquina, tentando melhorar a solução
do problema encontrada.

Sendo assim, as seguintes etapas deverão ser realizadas ao final deste trabalho:

\begin{enumerate}
  \item Fazer uma revisão bibliográfica sobre de métricas de software,
    especificamente de código fonte.
  \item Realizar um estudo empírico acerca das métricas de código fonte em
    questão.
  \item Modelar o problema de como interpretar e acompanhar essas métricas
    seguindo uma abordagem de aprendizado de máquina.
\end{enumerate}


\section{Organização do Trabalho}

Este trabalho está dividido em mais três (3) capítulos subsequentes. No capítulo
\ref{chap:metricas}, são apresentados conceitos teóricos sobre métricas de
software, passando por métricas de \textit{design} de código fonte até métricas de 
vulnerabilidade de código fonte. Nesse capítulo será possível entender um pouco mais 
sobre o objetivo das métricas de software e o porque da importância de se monitorar
métricas de vulnerabilidade. No capítulo \ref{estudodecaso}, é apresentado o
estudo de caso inicial realizado para este trabalho. Nele conterá a metodologia
utilizada, englobando hipóteses iniciais, as ferramentas utilizadas e o teste de
algumas hipóteses, além de uma análise qualitativa das métricas de
vulnerabilidade de código fonte. E por último, no capítulo
\ref{chap:consideracoes}, são feitas algumas considerações acerca do trabalho
feito até o momento, além de atividades a serem realizadas dentro do contexto de
um cronograma de trabalho.

