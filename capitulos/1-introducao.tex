\chapter{Introdução} \label{cap:introducao}

A medição é um processo auxiliar essencial para o desenvolvimento de software
com qualidade. É importante medir para entender e controlar processos, produtos 
e projetos \cite{ministerio_processo2012}. Medidas fornecem informações sobre 
objetos (processos, produtos e projetos) e eventos, tornando-os compreensíveis 
e controláveis \cite{fenton&pfleenger98}. Entretanto, muitas vezes, mede-se
simplesmente por medir \cite{ministerio_processo2012}, não justificando todo o 
esforço gasto, lembrando que sempre há esforço envolvido para se medir algo.

O cenário apresentado acima é o que geralmente ocorre quando as pessoas não estão
capacitadas para trabalhar no dado contexto. Quando as pessoas não estão
preparadas para trabalhar com determinadas métricas de software existem duas opções:
utilizam de maneira inadequada, não agregando valor ao processo de desenvolvimento
de software; ou apenas não utilizam, mesmo que sejam importantes para o processo de
desenvolvimento.

Muitas vezes, as métricas de código fonte se encaixam bem nessa situação, não são 
utilizadas dentro do processo de desenvolvimento porque não se sabe o que deve ser
feito com as mesmas e que decisões tomar. Quando utilizadas, são utilizadas apenas
as métricas de \textit{design} de código fonte, que são mais conhecidas, como as 
métricas orientadas a objetos de \emph{Chidamber e Kemerer} 
(\citeyear{chidamber&kemerer2002}), isso ocorre porque são métricas já consolidadas
e possuem várias diretrizes de como utiliza-las de maneira adequada. Novas classes 
de métricas de código fonte, como as métricas de vulnerabilidade de código fonte 
por exemplo, não são tão utilizadas no dia-a-dia do desenvolvimento de software, 
devido a falta de conhecimento de como interpreta-las e monitora-las.

É nesse contexto que se encaixa este trabalho, onde pessoas não utilizam ou utilizam
de maneira inadequada métricas de código fonte por falta de orientação de como 
interpreta-las e utiliza-las de maneira que agregue valor ao produto/negócio.

\section{Justificativa}

Sabendo que existe o problema dentro da Engenharia de Software de não saber como
interpretar e monitorar métricas de código fonte além das já conhecidas, que são 
as métricas de \textit{design} de código fonte, espera-se conseguir auxiliar os 
Engenheiros de Software com este trabalho. Tentando mostrar a melhor forma de 
interpretar e acompanhar esses novos tipos de métricas que estão surgindo nos últimos
tempos, muitos não as utilizam devido a falta de conhecimento sobre o que fazer
com essas informações advindas dessas novas métricas.

Relacionado as métricas de vulnerabilidade de código fonte que é a primeira
classe de métricas de código fonte trabalhada neste trabalho, existe a
questão da segurança dos softwares, que está bastante em voga nos últimos
tempos, o que reforça a necessidade de inserção de métricas de vulnerabilidade
de código fonte dentro do ciclo de desenvolvimento de software. O encontro
dessas vulnerabilidades ainda dentro do ciclo de desenvolvimento facilita a
correção das mesmas, além de evitar custos de possíveis futuras manutenções
corretivas. Mas para isso, é necessário saber interpretar esses valores
e acompanhar no decorrer do ciclo de desenvolvimento.

Existem várias iniciativas de pesquisas relacionadas a como associar
características de \textit{design} do código fonte com vulnerabilidades, como
nos trabalhos de \emph{Esposte e Bezerra} (\citeyear{arthur&carlos2014}) e
\emph{Alshmmari, Fidge e Corney} (\citeyear{alshammari2009}); entretanto, não
existem muitas iniciativas no mundo acadêmico de como interpretar e acompanhar
os valores dessas métricas de vulnerabilidade de código fonte, sendo essa uma
oportunidade ímpar de pesquisa.

Segundo um artigo divulgado pelo NIST, desenvolvido por \emph{Jansen}
(\citeyear{jansen2009}), o tema deste trabalho está totalmente alinhado com os
pensamentos da agência norte americana, onde a coleta de dados de projetos
existentes e a análise dos mesmos podem indicar padrões e informações
importantes para a medição da segurança de software, sendo essa apontada como
uma possível área de pesquisa.

Dentre outras coisas, este trabalho também servirá de insumo para próximos
trabalhos, já que, se possível, pretende-se indicar faixas aceitáveis de valores 
para as métricas de código fonte trabalhadas.

\section{Objetivos} \label{sec:objetivos}

Levando em consideração o que foi apresentado anteriormente, este trabalho tem
como objetivo principal determinar uma forma de interpretar e acompanhar
métricas de código fonte além das métricas de \textit{design}. Nesta
etapa do trabalho, o objetivo é encontrar uma forma de interpretar e monitorar,
especificamente, métricas de vulnerabilidade de código fonte.

No entanto, para isso, espera-se se aprofundar mais na teoria das métricas de
software e cada tipo de métricas de código fonte que deseja-se estudar, com o 
intuito de dominar o contexto que está sendo trabalhado e suas nuâncias.

Com o conhecimento teórico em mãos parte-se para o conhecimento prático,
empírico. Tendo assim, como objetivo, realizar estudos empíricos utilizando 
projetos de software livre, as métricas em questão e ferramentas para 
automatização desse processo também livre, para tentar responder a questão de 
como se deve interpretar e acompanhar os valores dessas métricas de código fonte.
Com esses estudos empíricos é passível de encontrar várias informações relevantes
que ainda não foram pensadas no âmbito teórico. Além da necessidade de prática
em um trabalho final de curso de Engenharia de Software.

Passando para as ferramentas utilizadas nos estudos empíricos, espera-se que
seja contribuição deste trabalho a manutenção evolutiva da ferramenta
\emph{Analizo}, com o que diz respeito a extração das métricas de
vulnerabilidade de código fonte. Esta funcionalidade já existe na ferramenta,
entretanto, necessita de alguns ajustes. Sendo o autor deste trabalho um dos
mantenedores da ferramenta e participante da equipe de desenvolvimento dessa
funcionalidade, espera-se contribuir com o projeto ao final, para que esta seja
a primeira ferramenta brasileira livre para extração desse tipo de métrica de 
código fonte.

Em outra etapa do trabalho será envolvido a questão do aprendizado de máquina, e
será um dos objetivos tentar modelar a problemática da interpretação e
acompanhamento das métricas de código fonte, além das de \textit{design},
seguindo essa abordagem. Sendo escopo deste trabalho apenas a modelagem do
problema seguindo a abordagem de aprendizado de máquina e não a implementação
desse modelo.

Em resumo, o principal objetivo de pesquisa deste trabalho é responder a
seguinte questão:

\begin{itemize}
  \item Como deve-se interpretar e acompanhar as métricas de código fonte além 
    das métricas de \textit{design}?
\end{itemize}


\section{Metodologia}

A metodologia utilizada para se atingir os objetivos deste trabalho foi: uma
revisão bibliográfica acerca do assunto de métricas de software, métricas de
\textit{design} de código fonte e métricas de vulnerabilidade de código fonte,
possivelmente serão realizados estudos sobre novos tipos de métricas de código
fonte que serão adicionados a este trabalho; além da realização de um estudo de 
caso, baseado em \emph{Meirelles} (\citeyear{meirelles2013}), para tentar entender 
melhor as métricas de vulnerabilidadede código fonte, como as mesmas se comportam 
e qual a melhor forma de monitora-las, tendo em vista que ainda não existe uma
quantidade considerável de referencial teórico acerca do assunto. Após a
consolidação de como interpretar e monitorar essas métricas, este trabalho
partirá para a abordagem de aprendizado de máquina, tentando melhorar a solução
do problema encontrada.

Sendo assim, as seguintes etapas deverão ser realizadas ao final deste trabalho:

\begin{enumerate}
  \item Fazer uma revisão bibliográfica sobre de métricas de software,
    especificamente de código fonte.
  \item Realizar um estudo empírico acerca das métricas de código fonte em
    questão.
  \item Modelar o problema de como interpretar e acompanhar essas métricas
    seguindo uma abordagem de aprendizado de máquina.
\end{enumerate}


\section{Organização do Trabalho}

Este trabalho está dividido em três (3) capítulos subsequentes. No capítulo
\ref{chap:metricas}, são apresentados conceitos teóricos sobre métricas de
software, passando por métricas de \textit{design} de código fonte até métricas de 
vulnerabilidade de código fonte. Nesse capítulo será possível entender um pouco mais 
sobre o objetivo das métricas de software e o porque da importância de se monitorar
métricas de vulnerabilidade. No capítulo \ref{estudodecaso}, é apresentado o
estudo de caso inicial realizado para este trabalho. Nele conterá a metodologia
utilizada, englobando hipóteses iniciais, as ferramentas utilizadas e o teste de
algumas hipóteses, além de uma análise qualitativa das métricas de
vulnerabilidade de código fonte. E por último, no capítulo
\ref{chap:consideracoes}, são feitas algumas considerações acerca do trabalho
feito até o momento, além de atividades a serem realizadas dentro do contexto de
um cronograma de trabalho.

