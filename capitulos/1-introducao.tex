\chapter*{Introdução} \label{introducao}
\addcontentsline{toc}{chapter}{Introdução}

\section*{Contextualização do Problema}

Em plena era da informação, vivemos em uma sociedade altamente dependente de computadores \cite{inclusao}, os quais utilizamos de forma direta ou indireta todos os dias: pagamos contas, utilizamos redes elétricas e de telefonia e sistemas de transporte como metrôs.  Falhas de software podem causar perdas irreparáveis, como roubo em massa de números de cartões de crédito\footnote{\url{http://bit.ly/1Tic4Yc}}, acidentes nucleares \cite{stuxnet} e perda em massa da privacidade dos cidadãos \cite{snowden}.

Apenas em 2014, foram registradas 7.937 vulnerabilidades\index{vulnerabilidade} no NVD (\textit{National Vulnerability Database})\footnote{\url{https://web.nvd.nist.gov/view/vuln/statistics-results?adv_search=true&cves=on}}, um banco de dados mantido pelo NIST\index{NIST} (\textit{National Institute of Standards and Technology}) que mantém registro de todas as vulnerabilidades\index{vulnerabilidade} de software publicamente conhecidas pela indústria e academia. Este foi o maior número de vulnerabilidades\index{vulnerabilidade} registrado em um único ano pelo NVD. Dentre as vulnerabilidades\index{vulnerabilidade} registradas estão o \textit{shellshock}\footnote{\url{https://cve.mitre.org/cgi-bin/cvename.cgi?name=CVE-2014-6271}} e o \textit{heartbleed} \cite{heartbleed}, duas das vulnerabilidades\index{vulnerabilidade} mais críticas já publicadas \cite{heartbleed}.

Dada a dependência da sociedade aos software que a tangem e o crescente número de vulnerabilidades\index{vulnerabilidade} encontradas nos mesmos, passa-se a buscar técnicas e métodos para aumentar a segurança de software \cite{sociedade}.

Embora acredite-se que buscar a melhoria da garantia de qualidade não seja suficiente para aumentar a segurança do software, segurança de software e qualidade de código estão fortemente relacionadas \cite{secure_programming}.

\section*{Questão de Pesquisa}

Avaliar ferramentas\index{ferramenta} de análise estática de código não é uma tarefa simples.
Inicialmente deve-se ter um escopo bem definido, como em \cite{harvard}, onde o autor propõe metodologia específica para a seleção de ferramentas\index{ferramenta} de análise estática para a detecção de defeitos\index{defeito} relacionados a \textit{Buffer Overflows}, dado que análise estática pode ser utilizada para contextos variados e específicos, como aquisição de métricas\index{métrica} de qualidade como apresentado por \cite{meirelles2013} e \cite{analizoartigo}, estilo, defeitos\index{defeito} relacionados à segurança, ou qualquer outro tipo de informação sobre o código que o desenvolvedor da ferramenta acreditar ser relevante.

Para \cite{secure_programming}, a ferramenta de análise estática pode se especializar em encontrar defeitos\index{defeito} específicos ou o máximo de defeitos\index{defeito} que seu nível de maturidade e técnicas de análise lhe permitirem encontrar. Além disso, os desenvolvedores da ferramenta podem optar por análises\index{análise} completas, corretas ou nenhuma das abordagens, buscando minimizar a quantidade de falsos positivos\index{falso positivo} e falsos negativos\index{falso negativo} gerados pela análise. Todos os fatores mencionados devem ser considerados no momento da seleção de ferramentas\index{ferramenta} de análise estática de código fonte, sendo assim, a questão de pesquisa deste trabalho é a seguinte: \textit{é possível delinear uma abordagem bem definida, que possa ser facilmente replicada, para que se avaliem ferramentas\index{ferramenta} de análise estática de código fonte voltadas à segurança?}.

Note que sempre que o termo \textit{análise estática de código-fonte} for mencionado ao longo deste trabalho, o autor se refere à análise estática de código-fonte voltada à segurança, a menos que qualquer outro fim esteja explícito junto ao termo.

\section*{Objetivos}
\subsection*{Objetivo Geral}

Definir uma abordagem replicável e de fácil compreensão para avaliação\index{avaliação} de ferramentas\index{ferramenta} de análise estática de segurança de código-fonte.

\subsection*{Objetivos Específicos}

Os objetivos específicos deste trabalho são:
\begin{itemize}
    \item analisar processos de avaliação\index{avaliação} ou seleção de ferramentas\index{ferramenta} de análise estática utilizados em estudos anteriores;
    \item definir processo para avaliação\index{avaliação} de ferramentas\index{ferramenta} de análise estática explicitando cada uma de suas atividades;
    \item validar o processo definido através de exemplos de uso.
\end{itemize}
\section*{Delimitação do Escopo}

Este trabalho busca propor uma abordagem replicável para análise estática de código-fonte, de modo que outras pessoas ou instituições, além das mencionadas em \nameref{fundamentacao_teorica}, possam ter alguma base para iniciar pesquisas relacionadas à ferramentas\index{ferramenta} de análise estática de segurança de código fonte. Este tipo de processo de avaliação\index{avaliação} deve ser melhorado ao longo de inúmeras iterações, de modo que o processo aqui apresentado trata-se de uma primeira versão do processo e não cabe à este trabalho apresentar etapas para melhoria do processo apresentado. Além disto, os estudos de caso apresentados ao final deste trabalho não visam avaliar de fato as ferramentas\index{ferramenta} apresentas, visam apenas validar a abordagem proposta.


\section*{Elementos do Trabalho}

Este trabalho foi dividido em 6 capítulos. O primeiro deles, \nameref{introducao}, contextualiza o problema a ser atacado e apresenta uma hipótese e escopo para este trabalho. Em seguida, a \nameref{fundamentacao_teorica} define análise estática e outros termos utilizados ao longo dos outros capítulos e termina apresentando trabalhos que apresentam algum elo com avaliação\index{avaliação} de ferramentas\index{ferramenta} ou análise estática de código fonte. Em \nameref{metodologia} apresentam-se as técnicas utilizadas para elaboração do trabalho e as decisões tomadas para a seleção das bibliografias relevantes para o mesmo. O Capítulo \nameref{metodologia_proposta} apresenta uma abordagem para avaliar ferramentas\index{ferramenta} de análise estática de segurança de código fonte. Tal abordagem é uma formalização e generalização de conceitos obtidos  a partir de experiências adquiridas ao longo de 12 meses de interação com o SAMATE\footnote{\url{http://samate.nist.gov}}, seu conjunto de dados de referência para garantia de software, SARD\footnote{\url{http://samate.nist.gov/SARD/}}, e a quinta edição de seu programa de Avaliação\index{avaliação} de Ferramentas\index{ferramenta} de Análise Estática, SATE V\footnote{\url{http://samate.nist.gov/SATE.html}}, bem como a participação do autor na avaliação\index{avaliação} das ferramentas\index{ferramenta} em uma das categorias do SATE V\footnote{\url{http://samate.nist.gov/SATE5OckhamCriteria.html}}. Em \nameref{exemplos_de_uso}, apresentam-se 2 estudos de caso onde se aplicam a abordagem aqui proposta: um estudo de avaliação\index{avaliação} de ferramentas\index{ferramenta} de licenças livres apontadas pelo NIST\index{NIST} e uma aplicação desta abordagem para realização das avaliações\index{avaliações} referentes à \textit{Ockham\index{ockham} Criteria}. Por fim, o Capítulo \nameref{conclusoes} encerra o trabalho, avaliando a abordagem apresentada através dos resultados obtidos nos estudos de caso e apontando trabalhos futuros relacionados à mesma e a outros tópicos relevantes.
