\chapter{Introdução} \label{introducao}

\section{Contextualização do Problema}

Em plena era da informação, vivemos em uma sociedade altamente dependente de computadores \cite{inclusao}, aos quais utilizamos de forma direta ou indireta todos os dias: pagamos contas, utilizamos redes elétricas e de telefonia e sistemas de transporte como metrôs.  Falhas de software podem causar perdas irreparáveis, como roubo em massa de números de cartões de crédito\footnote{\url{http://www.theguardian.com/technology/blog/2011/apr/29/playstation-network-hackers-credit-cards}}, acidentes nucleares \cite{stuxnet} e perda em massa da privacidade dos cidadãos \cite{snowden}.

Apenas em 2014, foram registradas 7.937 vulnerabilidades no NVD (National Vulnerability Database)\footnote{\url{https://web.nvd.nist.gov/view/vuln/statistics-results?adv_search=true&cves=on}}, um banco de dados mantido pelo NIST (National Institute of Standards and Technology) que mantém registro de todas as vulnerabilidades de software publicamente conhecidas pela indústria e academia. Este foi o maior número de vulnerabilidades registrado em um único ano pelo NVD. Dentre as vulnerabilidades registradas estão o \textit{shellshock}\footnote{\url{https://cve.mitre.org/cgi-bin/cvename.cgi?name=CVE-2014-6271}} e o \textit{heartbleed} \cite{heartbleed}, duas das vulnerabilidades mais críticas já publicadas \cite{heartbleed}.

Dada a dependência da sociedade aos software que a tangem e o crescente número de vulnerabilidades encontradas nos mesmos, passa-se a buscar técnicas e métodos para aumentar a segurança de software \cite{sociedade}.

Embora acredite-se que buscar a melhoria da garantia de qualidade não seja suficiente para aumentar a segurança do software, segurança de software e qualidade de código estão fortemente relacionadas \cite{secure_programming}.

\section{Hipótese}

Avaliar ferramentas de análise estática de código não é uma tarefa simples.
Inicialmente deve-se ter um escopo bem definido, como em \cite{harvard}, onde o autor propõe metodologia específica para a seleção de ferramentas de análise estática para a detecção de defeitos relacionados a \textit{Buffer Overflows}, dado que análise estática pode ser utilizada para contextos variados e específicos, como aquisição de métricas de qualidade como apresentado por \cite{meirelles2013} e \cite{analizoartigo}, estilo, defeitos relacionados à segurança, ou qualquer outro tipo de informação sobre o código que o desenvolvedor da ferramenta acreditar ser relevante.

Para \cite{secure_programming}, a ferramenta de análise estática pode se especializar em encontrar defeitos específicos ou o máximo de defeitos que seu nível de maturidade e técnicas de análise lhe permitirem encontrar. Além disso, os desenvolvedores da ferramenta podem optar por análises completas, corretas ou nenhuma das abordagens, buscando minimizar a quantidade de falsos positivos e falsos negativos gerados pela análise. Todos os fatores mencionados devem ser considerados no momento da seleção de ferramentas de análise estática de código fonte, sendo assim, este trabalho busca apresentar uma metodologia bem definida que possa ser facilmente replicada para que se avaliem ferramentas de análise estática de código fonte voltadas à segurança.

Note que sempre que o termo \textit{análise estática de código-fonte} for mencionado ao longo deste trabalho, o autor se refere à análise estática de código-fonte voltada à segurança a menos que qualquer outro fim esteja explícito junto ao termo.

\section{Objetivos}
\subsection{Objetivo Geral}

Definir metodologia replicável e de fácil compreensão para avaliação de ferramentas de análise estática de código-fonte com foco em segurança.

\subsection{Objetivos Específicos}

Analisar processos de avaliação ou seleção de ferramentas de análise estática utilizados em estudos anteriores;
Definir processo para avaliação de ferramentas de análise estática explicitando cada uma de suas atividades;
Validar o processo definido através de estudos de caso;

\section{Delimitação do Escopo}

O trabalho busca propor metodologia replicável para análise estática de código-fonte, de modo que outras pessoas ou instituições, além das mencionadas em \nameref{fundamentacao_teorica} possam ter alguma base para iniciar pesquisas relacionadas à ferramentas de análise estática com foco em segurança. Este tipo de processo de avaliação deve ser melhorado ao longo de inúmeras iterações, de modo que o processo aqui apresentado trata-se de uma primeira versão do processo e não cabe à este trabalho apresentar etapas para melhoria do processo apresentado. Além disto, os estudos de caso apresentados ao final deste trabalho não visam avaliar de fato as ferramentas apresentas, visam apenas validar a metodologia proposta.

\section{Outline do Trabalho}

Este trabalho foi dividido em 6 capítulos, o primeiro deles, \nameref{introducao}, contextualiza o problema a ser atacado e apresenta uma hipótese e escopo para este trabalho. Em seguida, o Capítulo \nameref{fundamentacao_teorica} define análise estática e outros termos utilizados ao longo dos outros capítulos e termina apresentando trabalhos que apresentam algum elo com avaliação de ferramentas ou análise estática de código fonte. Em \nameref{metodologia} apresentam-se as técnicas utilizadas para elaboração do trabalho e as decisões tomadas para a seleção das bibliografias relevantes para o mesmo. O Capítulo \nameref{metodologia_proposta} apresenta uma metodologia para avaliar ferramentas de análise estática com foco em segurança. Tal metodologia é uma formalização e generalização de conceitos obtidos  a partir de experiências adquiridas ao longo de 12 meses de interação com o SAMATE\footnote{\url{http://samate.nist.gov}}, seu conjunto de dados de referência para garantia de software, SARD\footnote{\url{http://samate.nist.gov/SARD/}}, e a quinta edição de seu programa de Avaliação de Ferramentas de Análise Estática, SATE V\footnote{\url{http://samate.nist.gov/SATE.html}}, bem como a participação do autor na avaliação das ferramentas em uma das categorias do SATE V\footnote{\url{http://samate.nist.gov/SATE5OckhamCriteria.html}}. Em \nameref{exemplos_de_uso}, apresentam-se 2 estudos de caso onde se aplicam a metodologia aqui proposta: um estudo de avaliação de ferramentas de licenças livres apontadas pelo NIST e uma aplicação desta metodologia para realização das avaliações referentes à Ockham Criteria. Por fim, o Capítulo \nameref{conclusoes} encerra o trabalho, avaliando a metodologia apresentada através dos resultados obtidos nos estudos de caso e apontando trabalhos futuros relacionados à mesma e à outros tópicos relevantes.
