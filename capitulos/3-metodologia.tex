\chapter{Metodologia}

Pretende-se nesse trabalho fazer uma análise qualitativa, utilizando distribuições estatísticas para determinar qual melhor
se associa a cada métrica de vulnerabilidade e possivelmente passando por percentis dos valores das mesmas. Para isso, serão
selecionados projetos de software livre para a realização de um estudo de caso especificado na seção \ref{estudodecaso}.
Serão coletadas métricas de vulnerabilidade de código fonte e as mesmas passarão por um processo de análise estatística
automatizado, utilizando \textit{scripts}\footnote{Disponível em https://gitlab.com/paulormm/tese} que foram desenvolvidos em 
\cite{meirelles2013}. Esse processo de análise estatística tentará identificar qual distribuição estatística melhor se encaixa 
nos dados referentes a determinada métrica, são utilizadas as seguintes distribuições estatísticas:

\begin{itemize}
  \item Pareto
  \item Pareto tipo 2
  \item Exponencial
  \item Gama
  \item Weibull
  \item Poison
\end{itemize}

Caso nenhuma das distribuições apresentadas consigam refletir os valores de uma determinada métrica de vulnerabilidade, os
\textit{scripts} tentarão encontrar uma distribuição estatística que melhor se adeque. Gráficos de cada distribuição serão
gerados como saída desse processo de análise estatística para melhor visualização dos dados.

De posse desses dados também espera-se determinar valores frequentes para as referidas métricas de vulnerabilidades de acordo
com a classe de sua aplicação.

Após a geração de todas essas informações será feita uma síntese e conclusão de como deve se dar o monitoramento de métricas de 
vulnerabilidade de código fonte dentro de projetos de software, assim podendo apontar trabalhos futuros.
