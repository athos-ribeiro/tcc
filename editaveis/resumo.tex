\begin{resumo}

Uma das técnicas para se reduzir o número de vulnerabilidades presentes em
software é através da detecção e resolução de defeitos presentes no código fonte
do mesmo. A detecção de tais defeitos pode ser automatizada com ferramentas de
análise estática de código fonte, quando estas têm um foco em aspectos de
segurança. O presente trabalho apresenta uma metodologia para que se avaliem
ferramentas de análise estática de código fonte com foco em segurança, desde a
seleção das ferramentas a serem avaliadas à apresentação dos resultados da
avaliação. Por fim, o trabalho apresenta exemplos de uso da metodologia
apresentada, onde se avaliam a capacidade de detecção de loops infinitos por
algumas ferramentas Livres e a corretude da análise de uma ferramenta que se
propõe a realizar análises corretas. 

 \vspace{\onelineskip}
    
 \noindent
 \textbf{Palavras-chaves}: Análise estática; Segurança; Código fonte; Vulnerabilidade.
\end{resumo}
