\begin{resumo}

Devido a constante evolução da Engenharia de Software, a cada dia surgem 
novas linguagens de programação, paradigmas de desenvolvimento, formas de
qualificar processos, entre outros. Com as métricas de código fonte não 
é diferente, com o passar do tempo surgem várias novas métricas e para
a utilização das mesmas vem a necessidade de se saber o por que de 
utiliza-las e como. Para a utilização de uma métrica de software qual for,
é necesário ter conhecimento sobre como realizar a coleta, cálculo, 
interpretação e análise para tomada de decisões. No contexto das métricas de
código fonte, a coleta e cálculo na maioria das vezes são automatizadas por
ferramentas, mas como interpreta-las e analisa-las de maneira correta no
decorrer do ciclo de desenvolvimento de software? Este trabalho visa tentar
auxiliar o Engenheiro de Software a interpretar e acompanhar métricas de 
código fonte além do \textit{design}, que já são bastante consolidadas. Tendo
como foco inicial as métricas de vulnerabilidade de código fonte.

 \vspace{\onelineskip}
    
 \noindent
 \textbf{Palavras-chaves}: Métricas; Análise; Código fonte; Vulnerabilidade.
\end{resumo}
